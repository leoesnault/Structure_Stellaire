\documentclass[a4paper,10pt]{article}
\usepackage[utf8]{inputenc}
\usepackage{hyperref}


\begin{document}

\title{Méthodes numériques : Structure Stellaire.}
\maketitle
\tableofcontents

\newpage

\section*{Notes}
Bouquin structure stellaire (Kippenhahn) :\\
\url{http://bookzz.org/book/2292709/51adf0}\\
Voire Chapitres I.1, I.2 ; II.10, II.11, II.12 ; IV.19, IV.21\\
\\
On s'intéresse à la structure mécanique d'étoiles pouvant être décrites par une équation d'état polytropique, soit les étoiles avec un gaz d'électrons dégénérés (type naine blanche, K fixé, n=3/2 si non relativiste, n=3 si relativiste), ou décrits par un gaz parfait + condition adiabatique (étoiles convectives très massives, K paramètre libre).\\
Un exemple de résolution des équations de Lane-Endem pour les naines blanches est donné ici : \url{http://userpages.irap.omp.eu/~rbelmont/mypage/numerique/naines_blanches.pdf}\\
%\url	{jila.colorado.edu/~pja/stars02/lecture12.ps}\\
\url{http://www.rfrench.org/astro/papers/P110-HET611-RobertFrench.pdf} Explications polytropes, Lane-Endem, exemple Soleil.\\
\url{http://www.unige.ch/ses/dsec/static/gilli/Teaching/MNE-Cours.pdf} Méthode de relaxation \\
\url{http://irfu.cea.fr/Projets/COAST/Derivation_of_stellareq.pdf} page 8 explication succinte méthode de relaxation dans le cadre de la structure stellaire polytropique\\


\section{Hypothèses}
\begin{list}{•}{}
\item Équilibre hydrostatique
\item Chimiquement homogène
\item Sans rotation propre
\item Équation d'état polytropique $ P = K \rho^{(1+\frac{1}{n})} $
\item Traitement Eulerien (variable indépendante : r)
\item Calcul à une dimension
\end{list}

\newpage
\section{Équations de structure stellaire et dédimensionnement}
Kippenhahn Ch.10, p89\\
Équation de conservation de la masse :
\begin{equation}
\frac{d m}{d r} = 4 \pi r^2 \rho
\end{equation}
\\
Équation d'équilibre hydrostatique :
\begin{equation}
\frac{d P}{d r} = - \frac{G m \rho}{r^2}
\end{equation}
\\
On a aussi l'équation de Poisson du potentiel gravitationnel\\
\begin{equation}
tatata
\end{equation}
\\

Kippenhahn Ch.20, p234 ou Ch.2 p 12\\
%On adimensionne les variables par rapport aux constantes caractéristiques suivantes :\\
%Le rayon de l'étoile $R$ , la pression centrale $P_c \approx \frac{2 G M^2}{\pi R^4}$ (estimation), la densité centrale $\rho_c =(\frac{P_c}{K})^{\frac{n}{n+1}}$, et la masse totale $M$. \\
On pose $q=\frac{m}{M}$ la masse adimensionnée, $z=A r$ le rayon adimensionné, et $w=(\frac{\rho}{\rho_c})^{\frac{1}{n}}$ la densité adimensionnée ; avec $A=\sqrt{\frac{4 \pi G}{(n+1) K \rho_c^{\frac{1-n}{n}}}}$ \\
\\
On part des équations (ref eqn Poisson) et (ref eqn équi hydro) et (ref eqn d'état).\\
Faire le développement.\\

En utilisant l'équation d'équilibre hydrostatique et l'équation de Poisson du potentiel gravitationnel, ainsi que l'équation d'état polytropique, on arrive à l'équation de Lane-Emden :\\
\begin{eqnarray}
\frac{1}{z^2}\frac{d}{dz}(z^2 \frac{d w}{dz}) + w^n=0\\
m(z)=4 \pi r^3 \rho_c \left(- \frac{1}{z} \frac{dw}{dz}\right)
\end{eqnarray}

\newpage
\section{Conditions aux limites}
Kippenhahn Ch.11, p93
\paragraph{Au centre}: \\
Rayon $r=0$ \\
Densité $\rho (r=0) = \rho_c$ \\
Variation de densité $\frac{\partial \rho}{\partial r}\vert_{r=0} = 0$ \\
Masse $m(r=0)=0$ \\
\\
Ce qui implique : \\
$z=0$ \\
$w=1$ \\
$\frac{dw}{dz}=0$ \\
$q=0$


\section{Discrétisation}
%On peut considérer une discrétisation sur N points, de $m=0$ à $M$.\\
%On aura $m=i*h$ avec $i=0 \dots N$, et $h=\frac{M}{N}$ le pas de masse.
%\subsection{Équation de r :}
%Si on utilise un schéma à droite :\\
%$\frac{r_{n+1} - r_{n} }{h}=\frac{1}{4 \pi r_{n}^2 \rho(r_{n})}$, soit\\
%\begin{equation}
%r_{n+1}=\frac{h}{4 \pi r_{n}^2 \rho(r_{n})}+r_n
%\end{equation}
%Variables de milne ?
%
%\subsection{Équation d'équilibre hydrostatique :}
%
%\begin{equation}
%\frac{d P}{d m} = - \frac{G m}{4 \pi r^4}
%\end{equation}

On considère une discrétisation de l'espace sur une grille de N+1 points, avec un pas $h$ régulier défini par $h=\frac{z_N}{N}$ en définissant une valeur de $z$  en N arbitraire.\\
On notera $X_i$ la valeur de la variable $X$ au point $i \in [0..N]$, la valeur $X_0$ correspondant à la valeur de $X$ au centre.\\
Partant du développement de Taylor-Young, il est possible d'obtenir des formules pour la discrétisation centrée ou excentrée autour d'un point $a$.\\
$f(a+h)=f(a)+\frac{h}{1!} f^{(1)} (a) + \frac{h}{2!} f^{(2)} (a) + \frac{h}{3!} f^{(3)} (a) + ...$\\
En prenant un pas $h$,$2h$, ... ou $-h$,$-2h$, ... et en faisant des combinaisons linéaires entre ces formules, on peut obtenir différents schémas de discrétisation.\\
On utilise ensuite ces schémas pour discrétiser les équations (ref eqn Lane-Emden et masse) :
\subsection{Lane-Emden}
\paragraph{pour $i=0$}: \\
On doit avoir la dérivée première de $w$ nulle à droite. On utilise donc un schéma excentré à droite à 3 points afin d'assurer cette condition. (justifier ?)\\
$f^{(1)} (a) = \frac{-3 w_i + 4 w_{i+1} - w_{i+2}}{2 h}=0$

\paragraph{pour $i=1..N-1$}: \\
Pour les dérivées d'ordres 1 et 2, on utilise un schéma centré à 3 points.\\
$f^{(1)} (a) = \frac{-w_{i-1}+w_{i+1}}{2 h}$\\
$f^{(2)} (a) = \frac{w_{i-1}-2 w_i+w_{i+1}}{h^2}$

\paragraph{pour $i=N$}: \\
Il réside un problème en $i=N$ par rapport au schéma précédent car un schéma centré fait appel à la valeur $w_{N+1}$ qui n'est dans notre cas pas définie. Il faut donc utiliser un schéma excentré à gauche, ici à 3 points pour les dérivées d'ordres 1 et 2.\\
$f^{(1)} (a) = \frac{w_{i-2}-4 w_{i-1}+3 w_i}{2 h}$\\
$f^{(2)} (a) = \frac{w_{i-2}-2 w_{i-1}+w_i}{h^2}$


\subsection{Équation de la masse}
De la même manière, on discrétise l'équation (ref) en utilisant respectivement un schéma excentré à droite à 3 points, un schéma centré à 3 points et un schéma excentré à gauche à 3 points.\\
$f^{(1)} (a) = \frac{-3 q_i + 4 q_{i+1} - q_{i+2}}{2 h}$\\
$f^{(1)} (a) = \frac{-q_{i-1}+q_{i+1}}{2 h}$\\
$f^{(1)} (a) = \frac{q_{i-2}-4 q_{i-1}+3 q_i}{2 h}$\\


\section{Méthode de relaxation}
Ayant défini les discrétisations au paragraphe précédent, on peut les insérer dans les équations (ref Lane-Emden et masse).


\section{Validation par comparaison à une méthode de référence}
Kippenhahn Ch.19 p216.\\
Il existe des solutions analytiques aux équations de structure stellaire de Lane-Endem pour les polytropes n=0, n=1 et n=5.\\
On compare les solutions trouvées avec ces solutions analytiques pour valider le modèle numérique.\\
Ces solutions sont données dans le tableau ci dessous :\\
\begin{tabular}{cc}
\hline
\\
n & w(z)\\
\\
\hline
\\
n=0 & $1 - \frac{1}{6} z^2$\\
\\
\hline
\\
n=1 & $\frac{\sin(z)}{z}$\\
\\
\hline
\\
n=5 & $\frac{1}{(1+\frac{z^2}{3})^{\frac{1}{2}}}$\\
\\
\hline
\end{tabular}

\section{Utilisation du programme}



\end{document}

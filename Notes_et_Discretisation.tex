\documentclass[a4paper,10pt]{article}
\usepackage[utf8]{inputenc}
\usepackage{hyperref}


\begin{document}

\title{Méthodes numériques : Structure Stellaire.}
\maketitle
\tableofcontents

\section*{Notes :}
Bouquin structure stellaire (Kippenhahn) :\\
\url{http://bookzz.org/book/2292709/51adf0}\\
Voire Chapitres I.1, I.2 ; II.10, II.11, II.12 ; IV.19, IV.21\\
\\
On s'intéresse à la structure mécanique d'étoiles pouvant être décrites par une équation d'état polytropique, soit les étoiles avec un gaz d'électrons dégénérés (type naine blanche, K fixé, n=3/2 si non relativiste, n=3 si relativiste), ou décrits par un gaz parfait + condition adiabatique (étoiles convectives très massives, K paramètre libre).\\


\newpage
\section{Hypothèses :}
- Équilibre hydrostatique\\
- Chimiquement homogène \\
- Sans rotation propre \\
- Équation d'état polytropique $ P = K \rho^{1+\frac{1}{n}} $ \\
- Traitement Lagrangien (variable indépendante : m)\\
- Calcul à une dimension

\section{Équations de structure stellaire :}
Kippenhahn Ch.10, p89
\subsection{Structure mécanique :}
%Équation d'état :
%\begin{equation}
%P=K \rho^\gamma
%\end{equation}
%\\
Équation du rayon :
\begin{equation}
\frac{d r}{d m} = \frac{1}{4 \pi r^2 \rho}
\end{equation}
\\
Équation d'équilibre hydrostatique : 
\begin{equation}
\frac{d P}{d m} = - \frac{G m}{4 \pi r^4}
\end{equation}
\subsection{Structure thermique et énergétique :}
\textbf{Équations découplées de la structure mécanique dans le cas d'une équation d'état polytropique. Ces équations ne sont donc pas considérées dans un premier temps.}\\
Équation de conservation de l'énergie :
\begin{equation}
\frac{d l}{d m}= \epsilon_{nuc}
\end{equation}
\\
Équation de transport de l'énergie :
\begin{equation}
\frac{d T}{d m}= - \frac{G m}{4 \pi r^4} \frac{T}{P} \bigtriangledown ~ avec ~ \bigtriangledown = \frac{d~ln T}{d~ln P}
\end{equation}

\section{Dédimensionnement des équations :}
Kippenhahn Ch.20, p234\\
On pose $q=\frac{m}{M}$ la masse adimensionnée.\\

\newpage
\section{Conditions aux limites :}
Kippenhahn Ch.11, p93
\paragraph{Au centre}: \\
%Flux d'énergie $l(m=0)=0$\\
Rayon $r(m=0)=0$\\
\\
Variation de pression $\frac{\partial P}{\partial m}\vert_{m=0} = 0$\\
Variation de densité $\frac{\partial \rho}{\partial m}\vert_{m=0} = 0$\\
%Variation de température $\frac{\partial T}{\partial m}\vert_{m=0} = 0$\\

\paragraph{En surface}:\\
Pression $P(m=M)=0$\\
Densité $\rho(m=M)=0$\\
%Température $T(m=M)=0$\\

\section{Passage en variables de Milne :}
Kippenhahn Ch.21, p243\\
$U=\frac{d~ln m}{d~ln r} $ et $V=-\frac{d~ln P}{d~lnr}$\\
Au centre, $U \rightarrow 3$ et $V \rightarrow 0$ .\\
En surface, $U$ devient très faible et $V$ augmente beaucoup.

\section{Discrétisation :}
%On peut considérer une discrétisation sur N points, de $m=0$ à $M$.\\
%On aura $m=i*h$ avec $i=0 \dots N$, et $h=\frac{M}{N}$ le pas de masse.
%\subsection{Équation de r :}
%Si on utilise un schéma à droite :\\
%$\frac{r_{n+1} - r_{n} }{h}=\frac{1}{4 \pi r_{n}^2 \rho(r_{n})}$, soit\\
%\begin{equation}
%r_{n+1}=\frac{h}{4 \pi r_{n}^2 \rho(r_{n})}+r_n
%\end{equation}
%Variables de milne ?
%
%\subsection{Équation d'équilibre hydrostatique :}
%
%\begin{equation}
%\frac{d P}{d m} = - \frac{G m}{4 \pi r^4}
%\end{equation}




\end{document}

\documentclass[a4paper,10pt]{article}
\usepackage[utf8]{inputenc}
\usepackage{hyperref}


\begin{document}

\title{Méthodes numériques : Structure Stellaire.}
\maketitle
\tableofcontents

\section*{Notes}
Bouquin structure stellaire (Kippenhahn) :\\
\url{http://bookzz.org/book/2292709/51adf0}\\
Voire Chapitres I.1, I.2 ; II.10, II.11, II.12 ; IV.19, IV.21\\
\\
On s'intéresse à la structure mécanique d'étoiles pouvant être décrites par une équation d'état polytropique, soit les étoiles avec un gaz d'électrons dégénérés (type naine blanche, K fixé, n=3/2 si non relativiste, n=3 si relativiste), ou décrits par un gaz parfait + condition adiabatique (étoiles convectives très massives, K paramètre libre).\\
Un exemple de résolution des équations de Lane-Endem pour les naines blanches est donné ici : \url{http://userpages.irap.omp.eu/~rbelmont/mypage/numerique/naines_blanches.pdf}

\newpage
\section{Hypothèses}
\begin{list}{•}{}
\item Équilibre hydrostatique
\item Chimiquement homogène
\item Sans rotation propre
\item Équation d'état polytropique $ P = K \rho^{(1+\frac{1}{n})} $
\item Traitement Lagrangien (variable indépendante : m)
\item Calcul à une dimension
\end{list}

\section{Équations de structure stellaire}
Kippenhahn Ch.10, p89
\subsection{Structure mécanique}
Équation de conservation de la masse :
\begin{equation}
\frac{d r}{d m} = \frac{1}{4 \pi r^2 \rho}
\end{equation}
\\
Équation d'équilibre hydrostatique : 
\begin{equation}
\frac{d P}{d m} = - \frac{G m}{4 \pi r^4}
\end{equation}
\subsection{Structure thermique et énergétique}
\textbf{Équations découplées de la structure mécanique dans le cas d'une équation d'état polytropique. Ces équations ne sont donc pas considérées dans un premier temps.}\\
Équation de conservation de l'énergie :
\begin{equation}
\frac{d l}{d m}= \epsilon_{nuc}
\end{equation}
\\
Équation de transport de l'énergie :
\begin{equation}
\frac{d T}{d m}= - \frac{G m}{4 \pi r^4} \frac{T}{P} \bigtriangledown ~ avec ~ \bigtriangledown = \frac{d~ln T}{d~ln P}
\end{equation}

\newpage

\section{Dédimensionnement des équations}
Kippenhahn Ch.20, p234 ou Ch.2 p 12\\
On adimensionne les variables par rapport aux constantes caractéristiques suivantes :\\
Le rayon de l'étoile $R$ , la pression centrale $P_c \approx \frac{2 G M^2}{\pi R^4}$ (estimation), la densité centrale $\rho_c =(\frac{P_c}{K})^{\frac{n}{n+1}}$, et la masse totale $M$. \\
On pose $\tilde{m}=\frac{m}{M}$ la masse adimensionnée, $\tilde{r}=\frac{r}{R}$ le rayon adimensionné, $\tilde{P}=\frac{P}{P_c}$ la pression adimensionnée et $\tilde{\rho}=\frac{\rho}{\rho_c}$ la densité adimensionnée. \\
\\
On obtient alors facilement les équations suivantes :
\begin{eqnarray}
\frac{d \tilde{r}}{d \tilde{m}}=\frac{1}{4 \pi \tilde{r}^2 \tilde{\rho}} \frac{M}{R^3 \rho_c} =\frac{1}{4 \pi \tilde{r}^2 \tilde{\rho}} A\\
\frac{d \tilde{P}}{d \tilde{m}}=-\frac{G}{4 \pi \tilde{r}^4} \frac{M}{R^4 P_c} =-\frac{G}{4 \pi \tilde{r}^4} B
\end{eqnarray}


\section{Conditions aux limites}
Kippenhahn Ch.11, p93
\paragraph{Au centre}: \\
%Flux d'énergie $l(m=0)=0$\\
Rayon $r(m=0)=0$\\
Variation de pression $\frac{\partial P}{\partial m}\vert_{m=0} = 0$
%Variation de température $\frac{\partial T}{\partial m}\vert_{m=0} = 0$\\

\paragraph{En surface}: \\
Pression $P(m=M)=0$
%Température $T(m=M)=0$\\

\section{Passage en variables de Milne}
Kippenhahn Ch.21, p243\\
$U=\frac{d~ln m}{d~ln r} $ et $V=-\frac{d~ln P}{d~lnr}$\\
Au centre, $U \rightarrow 3$ et $V \rightarrow 0$ .\\
En surface, $U$ devient très faible et $V$ augmente beaucoup.

\section{Discrétisation}
%On peut considérer une discrétisation sur N points, de $m=0$ à $M$.\\
%On aura $m=i*h$ avec $i=0 \dots N$, et $h=\frac{M}{N}$ le pas de masse.
%\subsection{Équation de r :}
%Si on utilise un schéma à droite :\\
%$\frac{r_{n+1} - r_{n} }{h}=\frac{1}{4 \pi r_{n}^2 \rho(r_{n})}$, soit\\
%\begin{equation}
%r_{n+1}=\frac{h}{4 \pi r_{n}^2 \rho(r_{n})}+r_n
%\end{equation}
%Variables de milne ?
%
%\subsection{Équation d'équilibre hydrostatique :}
%
%\begin{equation}
%\frac{d P}{d m} = - \frac{G m}{4 \pi r^4}
%\end{equation}


\section{Méthode de relaxation}



\section{Validation par comparaison à une méthode de référence}
Kippenhahn Ch.19 p216.\\
Il existe des solutions analytiques aux équations de structure stellaire de Lane-Endem pour les polytropes n=0, n=1 et n=5.\\
On compare les solutions trouvées avec ces solutions analytiques pour valider le modèle numérique.\\
Ces solutions sont données dans le tableau ci dessous :\\
\begin{tabular}{cc}
\hline 
\\
n & w(z)\\
\\
\hline 
\\
n=0 & $1 - \frac{1}{6} z^2$\\
\\
\hline 
\\
n=1 & $\frac{\sin(z)}{z}$\\
\\
\hline 
\\
n=5 & $\frac{1}{(1+\frac{z^2}{3})^{\frac{1}{2}}}$\\
\\
\hline 
\end{tabular}

\section{Utilisation du programme}



\end{document}
